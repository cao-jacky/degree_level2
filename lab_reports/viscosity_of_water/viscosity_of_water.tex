\documentclass[twocolumn]{revtex4}
\usepackage{graphics,graphicx,epsfig,ulem} 
\usepackage{amsmath}
\usepackage{multirow}
\usepackage{gensymb}
\usepackage{commath}
\usepackage{textcomp}
\newcommand{\squeezeup}{\vspace{-2.5mm}}

\begin{document}

\textheight=26.385cm
%Change textheight as the last resort...

\title{Determining the viscosity of water} 
 
 
\author{Jacky Cao, Room 205, Thursday, Lab Partners: Peter Dorey, Jon Pritchett \\ Date of experiment: 11/11/2016, Date of report: 20/11/2016}


\begin{abstract}              
 
asdkjasnkdnasd

\end{abstract}

\maketitle

\section{Introduction} 
\vspace{-2ex} 

Derived experimentally by Poiseuille in 1838 and Hagen in 1839 \cite{poiseuillehagen}, the volume flow rate $dV/dt$ of a fluid passing through a tube can be expressed as a function of the density of the fluid $\rho$, the value for acceleration due to gravity $g$, the height the fluid leaves the tube $h$, the radius $a$ and length $L$ of the tube, and the viscosity of the fluid $\eta$,

\begin{equation} 
\frac{dV}{dt}=\frac{\pi}{8}\frac{\rho gh}{\eta}\frac{a^4}{L}. 
\label{pohagen}
\end{equation}

Noting that the group $\rho gh$ can be collectively termed the pressure difference $\Delta P$ between the two ends of the tube \cite{collegephysics}. [[?? add something more to here]]

If we consider that a fluid is flowing through said [[?]] tube, it experiences both friction with the inner wall and internal friction within itself. The latter can be defined more readily as the viscosity of the fluid $\eta$, and this results in shear stress when two adjacent layers (laminas) of fluid move relative to each other. 

We find that as $\eta$ increases, the volume flow rate decreases, the shear stress between two laminas becomes greater and so restricts the movement of the fluid's molecules trying to flow through the tube. 

Generally we can say that the lamina flow streamlines are smooth, top layers sliding over other laminas without the system having any turbulent motion - this condition is required for equation [\ref{pohagen}] to be valid. 

Using a rearranged form of the relation derived by Hagen and Poiseuille it is possible to experimentally calculate a value for the viscosity of water.

\vspace{-3ex}
\section{Method} 
\vspace{-2ex}
A flow of water was created by fixing a capillary tube to a water tank. The tank was raised to an initial height arbitrary height above the work surface[?]. The tank was then filled up with water from heights 2cm to 16cm at 2cm intervals, these values were measured with the markings on the side of the tank. During this we had to ensure that we did not create parallax between the level of the water and the markings on the side. 

The water was then allowed to flow out of the tube for a period of 90s for each height of water, this time was measured using a digital stopwatch. As the water flowed out it was caught within a large beaker so that the mass and volume of it could be measured after the allotted time had passed. 

The mass was found by having the beaker already on a set of electronic scales and initially zeroed to account for the beaker's mass. The volume on the other hand required the water to be transferred from the beaker to a measuring cylinder, this was performed by using a pipette. Care was again taken so that there was no parallax and that no unaccounted water was left in the beaker. It was also necessary to clear the tube once air bubbles formed along the length of it, this reduced the flow rate of the water. The bubbles were cleared out using a long piece of copper wire. 

\begin{figure}[!h]
\begin{center}
\includegraphics[width=9cm]{fig1}
\caption[]{(a) A schematic of the experimental set-up used to collect the initial set of data. Entire set-up was placed on an optical bench, allowing accurate positioning of the lenses.
\\
(b) Modified set-up used in subsequent investigations - Newton's Rings Plate is moved, the screen is adjusted, and a Travelling Microscope is added.}
\label{fig:fig1}
\end{center}
\end{figure}

An initial density of water was calculated by taking four measurements of the volume and mass of the water, then calculating four values for $\rho_{water}$, and then averaging it so find a value. 

One set of data was taken for each of the three capillary tubes that were used. Each tube varied in their internal diameter. These values were calculated from multiple measurements taken using a travelling microscope along the horizontal axis.

The collected data was then applied to a least square fitting to create an initial linear model, then chi-squared analysis was performed to check if the model would hold true and if our data was accurate. After analysis three values of viscosity were calculated from the data. 

The length of the tube was measured using a ruler from one end to the other, so that the entire length was measured. [[???]]

\vspace{-3ex}
\section{Results}
\vspace{-2ex}

The volumetric flow rate of water is plotted against the varying height, as shown in Fig. \ref{fig:fig2}. This flow rate was calculated by dividing each value of the measured volume with the measured time period. Using this data, a value of viscosity could be calculated through a rearranged form of equation [\ref{pohagen}],

\begin{equation} 
\eta=\frac{\pi \rho g a^4 }{8 L m}, 
\label{r-pohagen}
\end{equation}

where $m$ is the gradient of the least squares regression line, calculated with the known data for $dV/dt$ and $h$.

From preliminary results taking, our calculated value for the density of water, which was used in further calculations is $1001 \pm 1 kg m^{-1}$.

The calculated values for the viscosity of water are shown in Table \ref{table:1} with which respective tube was used, the radius, the reduced $\chi^2$, and the Durbin-Watson statistic ($\mathcal{D}$) for each of those tubes. 

An average value for viscosity can thus be found to be $1.00 \pm 0.03 mPa\cdot{s}$, which [does/does not?] agree with the literature value \cite{crc}, $1.31m \pm ?? Pa\cdot{s}$. 

\vspace{-1ex}
\begin{figure}[!h]
\begin{center}
\includegraphics[width=9cm]{fig1-2}
\caption[]{The volume flow rate of water ($dV/dt$) as a function of height ($h$) for three capillary tubes of different diameter. The vertical error bars on $dV/dt$ are too small to be seen.}
\label{fig:fig2}
\end{center}
\end{figure}

\begin{table}[h!]
\centering
\begin{tabular}{ |c|c|c|c|c| } 
 \hline
 \textbf{Tube} & \textbf{Radius, a $[mm]$} & \textbf{$\boldsymbol{\eta_{water}}$ $[mPa\cdot{s}]$} & \textbf{$\boldsymbol{\chi^2_{\nu}}$} & $\mathcal{D}$\\ [0.5ex] 
 \hline\hline
 $Blue$ &$0.55\pm0.03$ & $1.0\pm0.2$ & 11.0 & 1.11\\ 
 $Red$ & $0.47\pm0.03$ & $1.1\pm0.3$ & 5.31 & 2.20\\
 $Black$ & $0.46\pm0.03$ & $1.0\pm0.3$ & 1.94 & 2.91\\
 
 \hline
\end{tabular}
\caption{Radius of the three tubes, their respective calculated value for $\eta_{water}$, the reduced $\chi^2$ value, and the calculated Durbin-Watson statistic, $\mathcal{D}$, is shown for each tube as well.}
\label{table:1}
\end{table}


\vspace{-3ex}
\section{Discussion}
\vspace{-2ex}
From an initial inspection of Fig. \ref{fig:fig2} we see that the variation of $dV/dt$ with $h$ for the blue and red tubes do not appear to fit a linear regression model, while the black tube does appear to have a more linear trend. We can also see that the majority of the points and their respective error bars for all three sets of data do not seem to be within the lines of best fit. This implies that either our initial assumption of linearity in our data is not valid, or that there were some defects in the experiment. 

We can further explore the validity of our model by looking at the calculated values for $\chi^2_{\nu}$. A reasonable fit to the data should see $\chi^2_{\nu}$ to be approximately one \cite{hughesandhayes}. We find that for our data, the blue and red tubes $\chi^2_{\nu}$s  are not close to one at all, but with the black tube, while not exactly one, it is a similar value. This affirms our initial assumption that the data for the black tube has more linearity, while we should consider changing the theoretical model for the other two tubes. 

Experimentally, we can consider that there were some limitations with how the data was collected. For the volume of the water, it was likely that some water was left within the pipette thus causing some of the data for $V$ to be not true [[change!!]]. 

When attatching the capillary tube on, the rubber seal was not always air tight so water did leak out! 

How can our experiment be limited? The volume - what changes the data, think of the setup  

Considerations of other factors were not taken into account such as the variation of the density of water with temperature. The temperature  The arising of random errors can not be considered as n

% should this paragraph be in here so early?
However, from Table \ref{table:1} we see that $\eta_{water}$ for each tube are in agreement within their experimental errors. While the model may not be correct, the values produced, which when averaged [are/are not] in agreement with literature values. 

The calculated values for reduced $\chi^2$ statistic further shows that the model chosen was not a good fit. If the data was a reasonable fit then a value for $\chi^2_{\nu}$ would be approximately one \cite{hughesandhayes}. however for the black tube we see that it has the closest value to 1, and if we also look at the normalised residuals as shown in Fig. [\ref{fig:fig2}] we see that the points are within the (-2,2) boundary limits. If we also look at Fig. [\ref{fig:fig2}] we see that for specifically the black tube, a linear least squares fitting model regression thing [?] is most suited for this.

The temperature of the environment also varied, a value was recorded during each measurement. Enviroment temperature has an effect. 

The experiment was limited by difficulties in measurement, the amount of data unable to be taken - more data could have been taken to account for the temperature change maybe. Also repeat measurements so the mean and standard error could be used instead of trying to calculate values yourself. More time needed so more data, in the time given, only one set of data was taken for each tube so that different flow rates could be considered. 

We can also draw that for varying radii sizes, the values are consistent? 

\vspace{-5ex}
\section{Conclusions}
\vspace{-2ex}
 
In conclusion, through experimentation it is possible to calculate values for the viscosity of water which are similar to 

\begin{thebibliography}{5}
\bibitem{poiseuillehagen}
	Salvatore P. Sutera and Richard Skalak
	\textit{The History of Poiseuille's Law}.
	Annu. Rev. Fluid Mech., 1993.
	
\bibitem{collegephysics}
	Raymond A. Serway, Chris Vuille, and Jerry S. Faughin
	\textit{College Physics, 8th Edition}.
	Brooks/Cole, Belmont, CA, USA, 2009.

\bibitem{youngandfreedman} 
	Hugh D. Young and Roger A. Freedman.
	\textit{University Physics with Modern Physics, 13th Edition}. 
	Pearson Education Limited, Essex, UK, 2015.
	
\bibitem{crc} 
	David R. Lie
	\textit{CRC Handbook of Chemistry and Physics, 84th Edition}. 
	CRC Press, Florida, USA, 2004.
	
\bibitem{hughesandhayes} 
	I. G. Hughes and T. P. A. Hase
	\textit{Measurements and their Uncertainties}. 
	Oxford University Press, Oxford, UK, 2010.
	
\end{thebibliography}
\clearpage

\vfill
\twocolumngrid
\vspace{-3ex}
\section*{Appendix}
\vspace{-2ex}

WIP

\clearpage
\end{document}